%2multibyte Version: 5.50.0.2960 CodePage: 65001
\documentclass[11pt, a4paper, reqno]{article}
%%%%%%%%%%%%%%%%%%%%%%%%%%%%%%%%%%%%%%%%%%%%%%%%%%%%%%%%%%%%%%%%%%%%%%%%%%%%%%%%%%%%%%%%%%%%%%%%%%%%%%%%%%%%%%%%%%%%%%%%%%%%%%%%%%%%%%%%%%%%%%%%%%%%%%%%%%%%%%%%%%%%%%%%%%%%%%%%%%%%%%%%%%%%%%%%%%%%%%%%%%%%%%%%%%%%%%%%%%%%%%%%%%%%%%%%%%%%%%%%%%%%%%%%%%%%
\usepackage{amssymb}
\usepackage{graphicx}
\usepackage{amsmath}

\setcounter{MaxMatrixCols}{10}
%TCIDATA{OutputFilter=LATEX.DLL}
%TCIDATA{Version=5.50.0.2960}
%TCIDATA{Codepage=65001}
%TCIDATA{<META NAME="SaveForMode" CONTENT="1">}
%TCIDATA{BibliographyScheme=Manual}
%TCIDATA{Created=Sat Aug 31 20:34:53 2002}
%TCIDATA{LastRevised=Wednesday, February 24, 2016 17:03:36}
%TCIDATA{<META NAME="GraphicsSave" CONTENT="32">}
%TCIDATA{<META NAME="DocumentShell" CONTENT="General\Blank Document">}
%TCIDATA{Language=American English}
%TCIDATA{CSTFile=Exam.cst}
%TCIDATA{PageSetup=72,72,72,72,0}
%TCIDATA{Counters=arabic,1}
%TCIDATA{AllPages=
%H=36
%F=36
%}


\newtheorem{theorem}{Theorem}
\newtheorem{acknowledgement}[theorem]{Acknowledgement}
\newtheorem{algorithm}[theorem]{Algorithm}
\newtheorem{axiom}[theorem]{Axiom}
\newtheorem{case}[theorem]{Case}
\newtheorem{claim}[theorem]{Claim}
\newtheorem{conclusion}[theorem]{Conclusion}
\newtheorem{condition}[theorem]{Condition}
\newtheorem{conjecture}[theorem]{Conjecture}
\newtheorem{corollary}[theorem]{Corollary}
\newtheorem{criterion}[theorem]{Criterion}
\newtheorem{definition}[theorem]{Definition}
\newtheorem{example}[theorem]{Example}
\newtheorem{exercise}[theorem]{Exercise}
\newtheorem{lemma}[theorem]{Lemma}
\newtheorem{notation}[theorem]{Notation}
\newtheorem{problem}[theorem]{Problem}
\newtheorem{proposition}[theorem]{Proposition}
\newtheorem{remark}[theorem]{Remark}
\newtheorem{solution}[theorem]{Solution}
\newtheorem{summary}[theorem]{Summary}
\newenvironment{proof}[1][Proof]{\textbf{#1.} }{\  \rule{0.5em}{0.5em}}
\usepackage{color}
\begin{document}


\begin{center}


\textbf{ECON 526 Advanced Macroeconomics, Yale University}, 
\\
 \textbf{Oliko Vardishvili}\bigskip \\
\textsc{Spring 2022 } \\

\underline{{\large Take Home Exam 1 }}\footnote{This take home is prepared for and adopted from  Arpad Abraham's lecture series. } \bigskip 
\end{center}

\noindent { \center (Upload a zip of code \& pdf files in Canvas) Note that for the final grade, I will ask you questions during lecture from your own solutions.}

\begin{description}
	\item[Problem 0] \textbf{Warm up:  Elastic Labor Supply} \newline
\end{description}
Now we will depart from the Aiyagari model with capital accumulation studied
in class by introducing elastic labor supply. Now the preferences take the following form:

\begin{equation}
 \dfrac{\big(c^{\nu}(1-l)^{1-{\nu}}\big)^{1-\mu}}{1-\mu}
\end{equation}


Take the following parameters: $\nu =0.374, \mu=4$.\footnote{Note that the values imply a relative risk aversion is equal to $2$, a standard value in macro literature. }
\textcolor{black}{ Use the following parameters:  $\beta=0.96$, $\alpha=0.36$, $\delta=0.08$, $\rho=0.6$, $\sigma_{\epsilon}=0.16$, such that $var(log(z_t))=0.04$. Besides, set $m=2.45$ to discretize shocks.}\footnote{$[prob,logs,invdist]=markovappr(rho,0.16,2.45,N)$ in Matlab}\\ 



%Utility: $u=\frac{c^{1-\gamma}}{1-\gamma}$, production function: $Y=AK^{\alpha}L^{1-\alpha}$
(a) Set up the dynamic programming problem and derive  the first-order conditions.\\ %compare the equilibrium capital with and without elastic labor supply
(b) Compute the steady state of the model using discrete value function iteration.\\ 


\begin{description}
\item[Problem 1] \textbf{Steady States} \newline
\end{description}
Now we again consider inelastic labor supply. We will depart from the Aiyagari model with capital accumulation studied
in class by introducing taxes in two very simple ways.

 Use the following parameters: $\gamma=0.5$, $\beta=0.96$, $\alpha=0.36$, $\delta=0.08$, $\rho=0.6$, $\sigma_{\epsilon}=0.16$, such that $var(z_t)=0.04$. Besides, set $m=2.45$ to discretize shocks. \vspace{0.2in} 

Utility: $u=\frac{c^{1-\gamma}}{1-\gamma}$, production function: $Y=AK^{\alpha}L^{1-\alpha}$

\begin{itemize}
\item Reform 1: Introducing a $15\%$ consumption tax. The tax revenues are rebated
equally across all agents of the economy.

\item Reform 2: Introducing a  capital income tax which generates the same revenues as the previous tax schedule (tax on $r k$, not on the principal). The tax revenues are rebated equally across all agents of the
economy again. [Hint: $\tau_k rk=Tr_c$]

\end{itemize}

\begin{enumerate}
\item Define the recursive competitive equilibrium in these cases. Note that
you have to add a government in the definition. Derive the Euler equations analytically and compare them with each other and with the basic model (without taxes).

\item Solve using some adjustments of the programs (with 3 solution methods: value function iteration, policy function iteration, and endogenous grid point methods provided in the class) for the steady state level of aggregate capital and the stationary decision rules and
distribution of agents for the two tax reforms. [Hint: Note that tax revenues and consequently the tax rebate is a function of aggregate capital (or the interest rate), so you have to make only small modifications.] 

\item Plot asset policy, consumption policy, and distributions and compare the smoothness of the functions across these solution methods. Explain which solution method approximates better the policy functions. Why?

\item Check how aggregate capital accumulation changes as a result of the
two tax reforms. Provide intuition in terms of insurance and output
efficiency.

\item Check how the distribution of agents across consumption levels and
asset levels changes due to the two reforms. You may want to use both
graphical representation and some statistics such as the Gini coefficient or
coefficient of variation.

\item Calculate the welfare effect of these tax reforms. Do it in two ways:
(i) use the aggregate social welfare and compare it across the three cases
(benchmark and two reforms);\ (ii) check also who benefits and who loses due
to these reforms. You can use the consumption equivalent measures for these welfare
comparisons.

\item Comment on what sense, these welfare comparisons across steady states
are meaningful or misleading.

\item Find an interesting quantitative question which you can answer using
this model and answer it.

\end{enumerate}
\begin{description}
\item[Problem 2] \textbf{Transitional dynamics} \newline
Now assume that at t=0 the tax schedule is in the steady state with consumption taxes (which you solved above. i.e., steady-state economy under Reform 1). At the beginning of period $1$, the government makes a surprise announcement that it abolishes consumption taxes and switches entirely to taxing a capital income (the second steady state you solved above).

\end{description}



\begin{enumerate}
\item Define the recursive competitive equilibrium with transitions.

\item Compute the transition path of the economy using  the algorithm provided in handout and the matlab code provided in tutorial. [try T=200]. Plot the transition paths of interest rate, wage, capital and welfare. Comment on the results you obtain.


\item a. Answer the following question: "How much do we need to change consumption of the agent in every state in the stationary equilibrium so that he'd be indifferent between living through the tax reform and living in the pre-reform economy?" Decompose welfare increase due to \textit{increased consumption level} and due to  \textit{reduced uncertainty}.

b. Now discuss which tax system is welfare improving taking into account the transitional dynamics (as opposed to steady state comparisons.). Also discuss which taxes are more distortionary.



\item What fraction of the overall population would support the reform? Compute and plot the measure of consumption equivalent variation.

\item Use your results to analyze which tax schedule is better in terms of efficiency and distribution and why.

\item Find an interesting quantitative question which you can answer using
this model and answer it.
\end{enumerate}
\newpage

%\noindent \textbf{Due Date: October 15, 11 a.m. (in class or by e-mail to me and Andreas).}





\end{document}
